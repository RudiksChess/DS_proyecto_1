\input{Configuraciones/paquetes}

%--------------------------

\begin{document}
 \thispagestyle{empty} 
    \begin{tabular}{p{15.5cm}}
    \begin{tabbing}
    \textbf{Universidad del Valle de Guatemala} \\\\
   \textbf{Estudiantes:} Augusto Alonso, David Cuellar, Rudik Rompich\\
   
   
   \textbf{Correos:}  \href{mailto:alo181085@uvg.edu.gt}{alo181085@uvg.edu.gt}, \href{mailto:cue18382@uvg.edu.gt}{cue18382@uvg.edu.gt},\href{mailto:rom19857@uvg.edu.gt}{rom19857@uvg.edu.gt}\\
   \textbf{Carnés:} 181085, 18382,19857
    \end{tabbing}
    \begin{center}
        CC3066 - Data Science I - Catedrático: Luis Furlan\\
        \today
    \end{center}\\
    \hline
    \\
    \end{tabular} 
    \vspace*{0.3cm} 
    \begin{center} 
    {\Large \bf  Code Book
} 
        \vspace{2mm}
    \end{center}
    \vspace{0.4cm}
%--------------------------


\section{Descripción general}
	\begin{figure}[H]
	\centering
	\includegraphics[scale=0.5]{Images/0}
\end{figure}

\newpage

\section{Descripción de variables}

\begin{variable}(codigo) 
	La variable código fue colocada arbitrariamente para identificar cada establecimiento educativo por un índice para un mejor manejo de los datos.
	\bigbreak 
	\textbf{Valores posibles.}
	\begin{itemize}
		\item 65021 índices distintos, representando a todos los centros educativos. 
	\end{itemize}
	\begin{figure}[H]
		\centering
		\includegraphics[scale=0.5]{Images/1}
	\end{figure}
\end{variable}

%-

\begin{variable}(var) 
	Content... 
	\bigbreak 
	\textbf{Valores posibles.}
	\begin{itemize}
		\item 
	\end{itemize}
	\begin{figure}[H]
		\centering
		\includegraphics[scale=0.5]{Images/2}
	\end{figure}
\end{variable}

%-

\begin{variable}(var) 
Content... 
\bigbreak 
\textbf{Valores posibles.}
\begin{itemize}
	\item 
\end{itemize}
\begin{figure}[H]
	\centering
	\includegraphics[scale=0.5]{Images/3}
\end{figure}
\end{variable}

%-

\begin{variable}(var) 
Content... 
\bigbreak 
\textbf{Valores posibles.}
\begin{itemize}
	\item 
\end{itemize}
\begin{figure}[H]
	\centering
	\includegraphics[scale=0.5]{Images/4}
\end{figure}
\end{variable}

%-

\begin{variable}(var) 
Content... 
\bigbreak 
\textbf{Valores posibles.}
\begin{itemize}
	\item 
\end{itemize}
\begin{figure}[H]
	\centering
	\includegraphics[scale=0.5]{Images/5}
\end{figure}
\end{variable}

%-

\begin{variable}(var) 
Content... 
\bigbreak 
\textbf{Valores posibles.}
\begin{itemize}
	\item 
\end{itemize}
\begin{figure}[H]
	\centering
	\includegraphics[scale=0.5]{Images/6}
\end{figure}
\end{variable}

%-

\begin{variable}(var) 
Content... 
\bigbreak 
\textbf{Valores posibles.}
\begin{itemize}
	\item 
\end{itemize}
\begin{figure}[H]
	\centering
	\includegraphics[scale=0.5]{Images/7}
\end{figure}
\end{variable}

%-

\begin{variable}(var) 
Content... 
\bigbreak 
\textbf{Valores posibles.}
\begin{itemize}
	\item 
\end{itemize}
\begin{figure}[H]
	\centering
	\includegraphics[scale=0.5]{Images/8}
\end{figure}
\end{variable}

%-

\begin{variable}(var) 
Content... 
\bigbreak 
\textbf{Valores posibles.}
\begin{itemize}
	\item 
\end{itemize}
\begin{figure}[H]
	\centering
	\includegraphics[scale=0.5]{Images/9}
\end{figure}
\end{variable}

%-

\begin{variable}(var) 
Content... 
\bigbreak 
\textbf{Valores posibles.}
\begin{itemize}
	\item 
\end{itemize}
\begin{figure}[H]
	\centering
	\includegraphics[scale=0.5]{Images/10}
\end{figure}
\end{variable}

%-

\begin{variable}(var) 
Content... 
\bigbreak 
\textbf{Valores posibles.}
\begin{itemize}
	\item 
\end{itemize}
\begin{figure}[H]
	\centering
	\includegraphics[scale=0.5]{Images/11}
\end{figure}
\end{variable}

%-

\begin{variable}(var) 
Content... 
\bigbreak 
\textbf{Valores posibles.}
\begin{itemize}
	\item 
\end{itemize}
\begin{figure}[H]
	\centering
	\includegraphics[scale=0.5]{Images/12}
\end{figure}
\end{variable}

%-

\begin{variable}(var) 
Content... 
\bigbreak 
\textbf{Valores posibles.}
\begin{itemize}
	\item 
\end{itemize}
\begin{figure}[H]
	\centering
	\includegraphics[scale=0.5]{Images/13}
\end{figure}
\end{variable}

%-

\begin{variable}(var) 
Content... 
\bigbreak 
\textbf{Valores posibles.}
\begin{itemize}
	\item 
\end{itemize}
\begin{figure}[H]
	\centering
	\includegraphics[scale=0.5]{Images/14}
\end{figure}
\end{variable}

%-

\begin{variable}(var) 
Content... 
\bigbreak 
\textbf{Valores posibles.}
\begin{itemize}
	\item 
\end{itemize}
\begin{figure}[H]
	\centering
	\includegraphics[scale=0.5]{Images/15}
\end{figure}
\end{variable}

%-

\begin{variable}(var) 
Content... 
\bigbreak 
\textbf{Valores posibles.}
\begin{itemize}
	\item 
\end{itemize}
\begin{figure}[H]
	\centering
	\includegraphics[scale=0.5]{Images/16}
\end{figure}
\end{variable}

%-

\begin{variable}(var) 
Content... 
\bigbreak 
\textbf{Valores posibles.}
\begin{itemize}
	\item 
\end{itemize}
\begin{figure}[H]
	\centering
	\includegraphics[scale=0.5]{Images/17}
\end{figure}
\end{variable}













%---------------------------
\bibliographystyle{apa}
\bibliography{referencias.bib}

\end{document}